\addchap{Hit Refresh}
\label{ch:hit-refresh}

{\flushright\largegreycapitals{Laurent B\'elanger}}
\smallskip

\initial{T}he guiding principle behind \emph{Hit Refresh} is apparent from its first
   chapter, in which Nadella introduces himself, not as a CEO, but simply
   as a person. Indeed, the first part of the book concerns itself almost
   entirely with detailing Nadella's background, from infancy to
   adulthood, and emphasising some crucial elements: his relationship with
   his parents, his early struggles in the United States, and the moment
   he learned that his first child would be born with cerebral palsy.

   To me, the intention of this first part is to prevent our perception of
   its author as little more than a walking job title. A glance at today's
   corporate industry shows us why: more often than not, executives live
   their titles, some even are their companies (think Bezos or Musk).
   Nadella's contention: that leaders should be understood, first and
   foremost, as human beings.

   This understanding certainly shouldn't be limited to leaders --
   everyone should attempt to understand each other. This might seem like
   an innocuous suggestion, but it represents a significant paradigm shift
   in corporate culture: the transition to praising understanding over
   perfection, and commending learning over unambiguous track records.

   Of course, as John Rossman remarked in his recent \href{https://www.goodreads.com/book/show/22393576-the-amazon-way}{book} on Amazon,
   leaders must be right -- a lot. Nadella doesn't dispute this, but
   proposes instead that the cornerstone of leadership should be empathy,
   and not performance. Performance, he argues, is heightened by a better
   understanding of the needs of others.

   This is not a novel idea: studies of leadership teams and executives
   have found that those with a higher emotional quotient \href{https://hbr.org/2015/04/measuring-the-return-on-character}{perform
   better} than their counterparts in similar situations. These studies
   also tell us, however, that executives tend to be least empathetic
   group -- and this is why I find \emph{Hit Refresh} significant: Nadella not
   only sees empathy as a necessary quality in leaders, but understands
   his own ability to empathise as the premise of his success at
   Microsoft, especially in his role as CEO. "Listening," he writes, "was
   the most important thing I accomplished each day."

   It is truly encouraging and inspiring to see a new generation of
   leaders, Nadella among them, working to reform the culture of their
   companies with this guiding principle in mind. Companies need to shift
   from Friedman's destructive \href{http://umich.edu/~thecore/doc/Friedman.pdf}{concept} of shareholder primacy toward
   creating long-term value for their employees, their investors, their
   customers, and most importantly, future generations.

   For more information on the book, see \href{https://www.goodreads.com/book/show/23258385-spaces-of-aid}{Hit Refresh} by Satya Nadella.
