\addchap{The Name of the Rose}
\label{ch:the-name-of-the-rose}

{\flushright\largegreycapitals{Dylan Springer}}
\smallskip

   \initial{T}he world is becoming a safer and more pleasant place for everyone in
   it. But never before have people felt so isolated and spiritually lost;
   and, with the advent of nuclear and biological weapons of war, the
   human race has never been so close to total extinction. We seem
   perpetually unable to decide whether these times are the greatest in
   human history or the worst, and we lament that all the scientific
   advances in the world have not made us happy, or sane, or peaceful.

   Perhaps what is needed is a little perspective: the perspective of
   people from a world totally different to our own. Umberto Eco's 1983
   international bestseller \emph{The Name of the Rose} offers a window into this
   world in the intriguing form of a 14th-century murder-mystery set in a
   remote Italian abbey. A key theme in his novel: that we should be
   skeptical of "prophets," and of "progress," in whatever guise they
   come.

   Today we are accustomed to thinking of "progress" as a linear, positive
   process. This was not always so. Indeed, until relatively recently,
   people thought that the world was in an unending state of decline and
   disorder, ever since the glory days of the Roman Republic, or perhaps
   even earlier, all the way back to the Garden of Eden. This line of
   thought is best encapsulated by Eco through the words of his narrator,
   the novice Adso of Melk:

     In the past men were handsome and great (now they are children and
     dwarves)... The young no longer want to study anything, learning is
     in decline, the whole world walks on its head, blind men lead others
     equally blind and cause them to plunge into the abyss... Everything
     is diverted from its proper course.

   Naturally, Adso's complaint is a gross exaggeration, and could easily
   have been uttered by any crotchety old man from the days of Aristotle
   to today, but it leads us to another key lesson of \emph{The Name of the
   Rose}: the importance of respecting the work done by one's predecessors.
   It has become common recently, and especially in the field of the
   social sciences, to discount the opinions and views of the old masters,
   either because we think they are archaic and useless or because we find
   them offensive in one way or another. We are so accustomed to
   looking forwards to new inventions and ideologies that we too often
   forget that the best and most useful wisdom is sometimes discovered by
   looking backwards. This is demonstrated in the way that the protagonist
   of the novel uses "the logic of Aristotle" and a newly-invented pair of
   spectacles to decipher encrypted messages and solve a series of brutal
   murders. The old and the new, working together.

   This protagonist, Brother William of Baskerville, provides an excellent
   role model for the young students and leaders of today. He stresses the
   importance of appreciating the value of tradition and old knowledge,
   keeping an open mind, of thinking critically, and expressing caution,
   moderation, and presence of mind throughout all aspects of life. "The
   Devil is the arrogance of the spirit, faith without smile, truth that
   is never seized by doubt," William says. In his own roundabout way, he
   teaches us how to be virtuous.

   You should read this book because of what it says about the "big"
   philosophical questions, for the role-model to be found in Brother
   William, and, finally, because it offers a window not only into a
   different period of time but also into a lost and sorely-needed way of
   thinking. With such guidance, we may be able to thrive in this terrible
   and wonderful era of human history.

   For more information on the book, see \href{The Name of the Rose} by
   Umberto Eco.

