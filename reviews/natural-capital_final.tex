\addchap{Natural Capital}
\label{ch:natural-capital}

   \initial{D}ieter Helm does an excellent job of lucidly explaining the concept of
   Natural Capital -- a subject unfamiliar to most readers, most of which
   myself. Helm's basic concept is to put an economic value on
   environmental goods and services such as clean water, forests,
   recreational spaces etc, such that governments would be compelled to
   account for them in their budgets. This would protect these
   environmental services and put a halt to the `tragedy of the commons'
   -- the abuse of natural resources as a result of lacking ownership and
   thus incentives for maintenance -- a process which is running rampant
   worldwide.

   The book begins with a mantra: explaining the mess we are in and why we
   have to change our approach to evaluating the environment. It then
   moves on to a more detailed discussion of Natural Capital. In this, it
   is unique among books concerned with the environment -- of which I have
   read many. Helm makes a convincing case for discrediting the "save
   everything" mantra, instead advocating a pragmatic approach of
   compromise, held up by the valuation of assets central to the concept
   Natural Capital. I challenge anyone to not be an advocate for Natural
   Capital after reading this book.

   Helm appealed to the environmentalist and the pragmatist in me,
   covering the complicated aspects of accounting for cross-generational
   costs, and explaining how his approach might be implemented in our
   economy. He further takes a very positive tone, detailing examples of
   where this process has been put to use, and how quickly and
   definitively it can effect progress.

   Helm slightly dodges the issue of politics and time scales. If it
   really is as simple as he makes it out to be, why isn't this approach
   to our environment already incorporated into our countries' political
   agenda? Nevertheless, Helm is an economist, and it is an unreasonable
   expectation for the book to cover this in any sort of detail.

   I would highly recommend this book to anyone interested in economics or
   climate change. The author's abundant knowledge and passion come
   through in his writing, and his argument is a thought provoking one: it
   compelled me to think about the subject in detail and accept his
   argument, and ultimately enlightened me in an area I knew very little
   about.

   For more information on the book, see [21]Natural Capital by Dieter
   Helm.

