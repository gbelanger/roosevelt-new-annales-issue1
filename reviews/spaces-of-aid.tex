\addchap{Spaces of Aid}
\label{ch:spaces-of-aid}

\initial{T}he first letter is special



   Every day, humanitarian aid workers around the world set off to play
   their part in lessening the radical and widespread inequality present
   in our modern, global society. Problematically, the humanitarian aid
   that the international community  provides is typically based on a set
   of idealised assumptions, what Smirl terms the "humanitarian
   imaginary." It is this set of assumptions, their connection to physical
   spaces of aid, and the disassociation that this creates that Smirl sets
   out to examine in her book.

   The physical setting of the aid environment is essential to the
   formation of relations in the conduct of humanitarian aid, and can be
   key to understanding the limitations of a mission. Yet since the advent
   of international aid, true analysis of its environment has remained
   overlooked.

   Spaces of Aid thoroughly examines spaces, objects, and environments, in
   the context of aid work. The first three chapters focus on the material
   and spatial environments of the international aid community. This is
   where the majority of the heavy theoretical discussion takes place. A
   very clear argument emerges here, which proves how essential an
   examination of material and spatial environments is to understanding
   every instance of aid intervention.

   Valuable examples -- the compound, the SUV, and the luxury hotel --
   ground the theoretical discussion of spatial relations in reality.
   Gated compounds, for instance, create distance between the physical
   environment of the aid workers and the local community, maintaining
   `hierarchical spacial divisions' reminiscent of colonialism. Similarly,
   the presence of security features perpetuates the conception that what
   exists outside the protective wall is dangerous. Ultimately, these
   approaches to the physical spaces of aid limit organisations' efficacy
   in delivering their goals: the physical segregation between aid workers
   and the communities that they serve only exacerbates their pre-existing
   socio-emotional distance from them.

   Following the theoretical discussion, the remaining two chapters make
   use of case studies to demonstrate the relationship between the
   physical and material, and the international aid worker and the local
   recipient. These case studies illustrate the manner in which
   preconceived conceptualisations of recipient communities determine the
   design of aid missions.

   Both Smirl's wealth of knowledge as a scholar, and her experience as a
   development professional make Spaces of Aid an impressive and
   ground-breaking analytical work. Its most important take-away: we must
   not consider spaces of aid as a tabula rasa. Every decision made by aid
   organisations originates from an outsider's conceptualisation of what
   the "other" needs and must be recognised as such. We need to eliminate
   barriers between aid workers and the communities they serve -- both
   physical and perceived -- and in so doing heighten organisations'
   understanding of these communities, an understanding which can be
   translated into better-designed, more effective humanitarian missions.


   For more information on the book see [21]Spaces of Aid by Lisa Smirl.

