   #[1]RSS Feed

   [2]The Roosevelt Club
   (BUTTON)
   [3]Cart 0

   [4]Name [5]Mission (BUTTON) Publications [6]People

   [7]Join
   (BUTTON) Back [8]New Annales [9]Quick-Takes [10]Reviews
   (BUTTON)

   ____________________ (BUTTON)

   [11]Cart 0

   [12]Name[13]Mission [14]Publications [15]New Annales [16]Quick-Takes
   [17]Reviews [18]People
   [19]The Roosevelt Club
   virtus

   [20]Join

Designer Babies


   Human birth, in the way most people conceptualise it, entails a natural
   life process of one human giving birth to another. However, as of 2018,
   human birth can no longer be restricted to this definition. With
   processes such as in vitro fertilisation (IVF), babies can be grown
   outside of the mother's womb. With the rise of "test tube babies" has
   come the possibility for parents to handpick a child's genes. Though
   socioeconomic factors put limitations on the number of parents with
   access to the genetic modification of their children, the moral and
   political implications of genetic modification render it an important
   topic. With genetic modification, parents can control the sex of a
   child, their physical characteristics, and in some cases, even enhance
   their intelligence.

   The advancement of bio-technology seems to be developing at such a
   rapid pace that society has neglected to consider the ethical
   ramifications of this new technology. Both state governments, like the
   United States and the United Kingdom, and supranational organisations
   like the European Union are struggling to create policies and laws to
   protect and contain the potential consequences of these genetically
   modified humans. This article will attempt to outline the ethical
   consequences these international governments face by legalising the
   unadulterated genetic modification of humans. The argument against
   genetic modification will be split into three concise sections: the
   argument for freedom, the slippery slope argument, and the argument for
   equality.

   To begin, it is important to set aside the argument of whether or not
   genetically modifying babies is ethical, and for a moment consider a
   larger question of freedom. Though this might seem like a strategic
   diversion by avoiding the harder question in favour of an existential
   one, this question is pertinent to the argument. Freedom, in the way
   most people understand it, is the ability to think, feel, and act as
   one pleases. To be free is to have the ability steer one's life in
   whatever direction one chooses. However, this freedom, this ability to
   create one's own destiny, is compromised by genetic modification.

   Prior to genetic modification, children are born with varying
   abilities. Some are athletically gifted, others blessed with musical
   talent, and some conceived with unbelievable minds. These unique gifts
   are not predetermined but naturally inherited. With parents choosing
   their children's genetics, the life of these babies encompasses a
   sudden predetermination that has never been dealt with before. To
   illustrate, if a child is genetically modified to be tall, fast, and
   slender, with the object of their becoming an NFL quarterback, an
   individual's personal autonomy is compromised. While most parents hope
   their children are athletic, often forcing them into sports camps and
   primary school teams, altering the genetic makeup of a child is a
   different measure -- it inherently makes the playing field uneven.

   In addition to tampering with the innate abilities of humans, genetic
   modification jeopardises the concept of "free will".  While some will
   argue that the genetically modified children are still free to make
   their own decisions, this argument does not successfully dismiss the
   issue. To illustrate, Michael Jordan is universally agreed to be an
   accomplished basketball player. However, Jordan was not merely born
   with his endurance, speed, and precision. He worked hard to earn his
   title. Though parents have the ability to force their children to play
   sports, there is a difference between being forced to play sports and
   being genetically engineered to play sports. Children can resist their
   parents and refuse to play a sport, however, children do not have the
   ability to refuse genetic modification. The differentiation between
   being `forced' and being `engineered' creates the issue of a child's
   free will.

   If designer children are merely engineered to have these talents, the
   definition of hard work, skill, and success becomes obscured. Without
   genetic modification, Jordan had to demonstrate agency and a strong
   work-ethic, devoting his time and money to become professional
   basketball player. If Jordan was designed to be an excellent basketball
   player, his success and accomplishments would seem less commendable.
   While Jordan chose to be a basketball player, the genetically modified
   child did not. Critics might respond by saying that even with these
   genetic advantages, one would nevertheless have to possess a strong
   work ethic. Though this is probable, it does not defeat the argument
   that, in terms of athleticism, the genetically engineered child will
   still have an innate advantage.

   As wisely stated by ethicist Michael Sandler, "rather than employ our
   new genetic power to straighten the crooked timber of humanity, we
   should do what we can to create social and political arrangements more
   hospitable to the gifts and limitations of imperfect human beings."
   Instead of creating the "perfect NFL player", society should recognise
   the imperfect player who works and trains to overcome obstacles,
   showing a better command over their life success than a human that is
   merely designed to be perfect. Perfection eliminates failure, and
   without failure, the meaning of success begs for a revised definition.

   With human's individual autonomy at risk, it is also important to
   address what the legalisation of "designer babies" could mean in a
   legal sense for international governments such as those of the European
   Union. In January  [21]EU Advocate General Michal Bobek attempted to
   begin the discussion regarding how these genetic technologies should be
   regulated. Many EU representatives feel that genetic modification in
   humans should be legalised for the sole reason of preventing various
   diseases. For example, by having access to the genes of a foetus,
   scientists have the ability to eliminate predisposition to illnesses
   such as cancer, diabetes, and even blindness. Though scientists do
   possess the ability to prevent these illnesses, it does not justify
   them using these technologies without assessing the implications of
   their actions.

   Diseases such as diabetes and various cancerous cells are recognised as
   detrimental ailments that have robbed humans of their lives for
   countless years. However, while eliminating these genes may have
   beneficial short term effects, its long term consequences massively
   outweigh its benefits. While cancer and diabetes are universally deemed
   as "bad" sicknesses, handicaps such as blindness or deafness may incur
   more debate.

   For example, in 2008 a deaf couple in the United Kingdom, Tomato and
   her partner Paula, wished to have a deaf child. Their first child was
   coincidentally born deaf. Preparing for their second child, Paula and
   Tomato wanted to use IVF to produce another deaf child. However,
   according to [22]parliament's clause 14/4/9, the selection of a hearing
   child through IVF is permitted, but, embryos found to have deafness
   genes will be automatically discarded. The case of Tomato and Paula
   showed the United Kingdom's implicit preference for individuals who
   have normal hearing capabilities. This offended the international deaf
   community causing those such as Steve Emery, a sign language expert at
   Heriot Watt University to speak out. Emery publicly stated, "This
   clause sends out a clear and direct message that the UK government
   thinks deaf people are better off not being born." With handicaps such
   as deafness, it seems a moral overstep to allow the government to
   decide what traits are beneficial for survival and what traits should
   be seen as a malfunction in need of correction. If the EU Health
   Council decides to legalise genetic modification in respect to
   correcting illnesses and genetic predispositions, normal traits such as
   hearing and sight will be labelled as superior, those with atypical
   characteristics to be deemed as lesser. This will create a whole new
   class system within Europe, those who were groomed to be genetically
   superior, and those who have atypical human functions.

   The slippery slope that occurs when genetic modification is legalised
   for "health" reasons shows that even an inherently good action can have
   substantial consequences. While a world free of cancer and diabetes
   appears a utopia, with closer examination, this idealistic visions
   fades away into a hellion dystopia. If the ability to eliminate
   precancerous genes became available, it would only be accessible to the
   wealthy. This would create a class system of those who are genetically
   "superior" and those who either could not afford to be "designed" at
   birth or those who chose not to be. This reality would be eerily
   similar to the warnings of movies like Gattaca or television programs
   such as Black Mirror. Though an extreme comparison, Hitler's eugenics
   vision, of a "perfect race" would seem to be similar to the
   "genetically perfect" Europeans, Americans, or British, that would come
   from a genetically modified DNA.

   As philosopher Immanuel Kant once said, "human beings are ends in
   themselves, worthy of respect."  To tamper with the genes of a future
   human, to predetermine their characteristics without weighing the
   political, legal, and ethical consequences of these actions, seems
   nothing less than a neglect of human dignity.  In the words of Michael
   Sandler, "to change our nature to fit the world, rather than the other
   way around, is actually the deepest form of disempowerment." We as
   humans have the responsibility to work through adversity, seeing the
   beauty in imperfection. Great ideas and innovations stem from atypical
   individuals such as Albert Einstein or Vincent Van Gogh. Einstein did
   not read or speak until he was five, being diagnosed with severe
   autism. Van Gogh had depression. These two men changed the course of
   history, not despite their imperfections, but because of them: Einstein
   with his mathematical break-throughs and Van Gogh with his
   transformation of modern art. Simply put, imperfection within humans
   should be championed, not treated as a problem begging a solution.

   [23]Zoe SpirgelSeptember 26, 2018

   Previous

The Ethics of Self-Driving Cars

   Laurent B�langerJanuary 9, 2019ethics, self-driving cars, tesla, moral
   machine

   [24]Contact
   [25]Member resources

   [26]Home[27]Name[28]Mission[29]Publications[30]People

References

   Visible links:
   1. https://rooseveltclub.co.uk/quick-takes?format=RSS
   2. https://rooseveltclub.co.uk/
   3. https://rooseveltclub.co.uk/cart
   4. https://rooseveltclub.co.uk/about/
   5. https://rooseveltclub.co.uk/services/
   6. https://rooseveltclub.co.uk/testimonials/
   7. https://rooseveltclub.co.uk/lets-eat/
   8. https://rooseveltclub.co.uk/new-annales-2/
   9. https://rooseveltclub.co.uk/quick-takes-2/
  10. https://rooseveltclub.co.uk/reviews-2/
  11. https://rooseveltclub.co.uk/cart
  12. https://rooseveltclub.co.uk/about/
  13. https://rooseveltclub.co.uk/services/
  14. https://rooseveltclub.co.uk/publications-1/
  15. https://rooseveltclub.co.uk/new-annales-2/
  16. https://rooseveltclub.co.uk/quick-takes-2/
  17. https://rooseveltclub.co.uk/reviews-2/
  18. https://rooseveltclub.co.uk/testimonials/
  19. https://rooseveltclub.co.uk/
  20. https://rooseveltclub.co.uk/lets-eat/
  21. https://allianceforscience.cornell.edu/blog/2018/01/european-union-opinion-on-gene-editing-insightful-or-missed-opportunity/
  22. https://www.theguardian.com/science/2008/mar/09/genetics.medicalresearch
  23. https://rooseveltclub.co.uk/quick-takes/?author=5b9b82ff758d46874be441d6
  24. https://rooseveltclub.co.uk/contact-us
  25. https://rooseveltclub.co.uk/members
  26. https://rooseveltclub.co.uk/home
  27. https://rooseveltclub.co.uk/about
  28. https://rooseveltclub.co.uk/services
  29. https://rooseveltclub.co.uk/menu
  30. https://rooseveltclub.co.uk/testimonials

   Hidden links:
  32. https://rooseveltclub.co.uk/search
  33. http://instagram.com/therooseveltclub
  34. http://www.facebook.com/354310218289537
  35. https://rooseveltclub.co.uk/quick-takes/the-ethics-of-self-driving-cars
