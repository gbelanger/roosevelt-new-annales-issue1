   #[1]RSS Feed

   [2]The Roosevelt Club
   (BUTTON)
   [3]Cart 0

   [4]Name [5]Mission (BUTTON) Publications [6]People

   [7]Join
   (BUTTON) Back [8]New Annales [9]Quick-Takes [10]Reviews
   (BUTTON)

   ____________________ (BUTTON)

   [11]Cart 0

   [12]Name[13]Mission [14]Publications [15]New Annales [16]Quick-Takes
   [17]Reviews [18]People
   [19]The Roosevelt Club
   virtus

   [20]Join

Cannabis Legalisation


   Currently, the black market for cannabis in the UK is worth �2.6bn;
   cannabis is sold to over 3 million people a year. The Win-Win-Win of
   legalising cannabis, [21]as it was described by the Institute of
   Economic Affairs, has interesting effects on the political and economic
   landscape. Just like tobacco and alcohol, if cannabis became legalised,
   it would be distributed and sold in a regulated way. This would allow
   it to be taxed. Is this the biggest issue in party politics?

   Currently cannabis is sold and distributed through a black market of
   organised crime. The strength of the drug is not regulated, and neither
   is money generated through its sale. The consequences of this are felt
   by the taxpayer in higher policing costs and costs relating to mental
   health. Generally, there seem to be two schools of thought:
   decimalisation and legalisation. The former would treat possession of
   the drug as something as minor as a parking offence, while legalisation
   would allow free access and usage.

   A vision for the future could make buying cannabis as easy as buying a
   pint of beer, with similar age restrictions and tax levels. Proponents
   of this model talk about the benefits to the end user. With regulated
   supply and quality, the drug would be of a higher quality than many
   people can get off a street corner. The reduced demand on policing and
   hospitals could be as large as �300bn per year. The political
   motivations for this seem clear. With the younger voters becoming
   increasingly disenchanted with politicians, and in particular the
   Conservative Party, I was sure that this would be a vote winner.

   But legalisation is problematic. Firstly there are political issues. A
   BMG [22]research poll from 2018 found that only 51% of people supported
   legalisation understood as `making it as accessible as tobacco and
   alcohol.' Results from a You Gov [23]survey has revealed that a small
   majority of people would back a relaxation of the law. This small
   majority can hardly justify a change in the law, and even if it did, it
   would struggle to break the inertia of politics to be done quickly.

   But let us put the political issues aside for a moment and consider the
   tax model for this proposal. Creating the right level of taxation is
   important. Too high, and you do not discourage the black-market; too
   low, and it does not generate revenue to make it worthwhile. Across the
   Atlantic, in the states of Colorado, Washington and California, which
   have legalised cannabis, black markets have flourished due to the high
   prices caused by excessive taxation. In the IEA 2018 report, they
   propose a VAT plus an excise tax, which they estimate would generate
   �690m directly, and would save public services �300m a year. An NGO[24]
   has estimated the potential benefits of legalisation much higher at
   �3.5bn.

   It is hard to find a rational argument to support the existing
   legislation. While legalisation seems to promote a healthier form of
   the drug, it wouldn't necessarily increase consumption. That is putting
   aside the wider societal benefits of legalisation which have already
   been discussed. This view is echoed by William Hauge, who [25]said that
   the current law is "inappropriate, ineffective and utterly out of
   date." But just because it is rational doesn't mean it will be
   supported by votes, as the evidence shows. Thus, while a change in the
   law is likely, there is considerable political inertia against it.

   [26]Freddie KellettJanuary 23, 2019[27]cannabis, [28]legalisation,
   [29]UK, [30]politics, [31]economics

   Next

The Ethics of Self-Driving Cars

   Laurent B�langerJanuary 9, 2019ethics, self-driving cars, tesla, moral
   machine

   [32]Contact
   [33]Member resources

   [34]Home[35]Name[36]Mission[37]Publications[38]People

References

   Visible links:
   1. https://rooseveltclub.co.uk/quick-takes?format=RSS
   2. https://rooseveltclub.co.uk/
   3. https://rooseveltclub.co.uk/cart
   4. https://rooseveltclub.co.uk/about/
   5. https://rooseveltclub.co.uk/services/
   6. https://rooseveltclub.co.uk/testimonials/
   7. https://rooseveltclub.co.uk/lets-eat/
   8. https://rooseveltclub.co.uk/new-annales-2/
   9. https://rooseveltclub.co.uk/quick-takes-2/
  10. https://rooseveltclub.co.uk/reviews-2/
  11. https://rooseveltclub.co.uk/cart
  12. https://rooseveltclub.co.uk/about/
  13. https://rooseveltclub.co.uk/services/
  14. https://rooseveltclub.co.uk/publications-1/
  15. https://rooseveltclub.co.uk/new-annales-2/
  16. https://rooseveltclub.co.uk/quick-takes-2/
  17. https://rooseveltclub.co.uk/reviews-2/
  18. https://rooseveltclub.co.uk/testimonials/
  19. https://rooseveltclub.co.uk/
  20. https://rooseveltclub.co.uk/lets-eat/
  21. https://www.theguardian.com/society/2018/jun/29/legalise-cannabis-in-uk-institute-for-economic-affairs
  22. https://www.independent.co.uk/news/uk/politics/cannabis-decriminalisation-oil-uk-cigarettes-alcohol-bmg-research-poll-a8445631.html
  23. https://yougov.co.uk/news/2018/05/30/majority-now-support-liberalising-policy-towards-c/
  24. https://www.independent.co.uk/voices/decriminalise-marijuana-skunk-cannabis-health-poverty-action-a8381646.html
  25. https://www.telegraph.co.uk/news/2018/06/18/war-cannabis-has-failed-utterly-tories-should-consider-new-approach/
  26. https://rooseveltclub.co.uk/quick-takes/?author=5b572e761ae6cf85cd48296f
  27. https://rooseveltclub.co.uk/quick-takes/?tag=cannabis
  28. https://rooseveltclub.co.uk/quick-takes/?tag=legalisation
  29. https://rooseveltclub.co.uk/quick-takes/?tag=UK
  30. https://rooseveltclub.co.uk/quick-takes/?tag=politics
  31. https://rooseveltclub.co.uk/quick-takes/?tag=economics
  32. https://rooseveltclub.co.uk/contact-us
  33. https://rooseveltclub.co.uk/members
  34. https://rooseveltclub.co.uk/home
  35. https://rooseveltclub.co.uk/about
  36. https://rooseveltclub.co.uk/services
  37. https://rooseveltclub.co.uk/menu
  38. https://rooseveltclub.co.uk/testimonials

   Hidden links:
  40. https://rooseveltclub.co.uk/search
  41. http://instagram.com/therooseveltclub
  42. http://www.facebook.com/354310218289537
  43. https://rooseveltclub.co.uk/quick-takes/the-ethics-of-self-driving-cars
