\addchap{The Ethics of Self-Driving Cars}
\label{ch:the-ethics-of-self-driving-cars}

{\flushright\largegreycapitals{Laurent B\'elanger}}
\smallskip

   \initial{T}he advent of self-driving cars has had mixed results: on the one hand,
   it is important technological progress that bears witness to our
   enduring ingenuity and the possibility of ambitious change; on the
   other, it has served as a testament to how wholly unprepared the global
   community is to face widespread technological change whose impact on
   human activity is uncertain.

   One issue that has garnered some attention in the recent debate over
   self-driving cars and defining their role within our societal framework
   is the question of how they should `react' to critical situations in
   which the occurrence of an accident is almost certain.

   An excellent \href{http://www.thesaint-online.com/2018/11/the-moral-complexities-of-the-self-driving-car/}{article} in \emph{The Saint} by Mr. Martin George provides a
   good elucidation of the question, and the different considerations that
   it implies. The central consideration in this question seems to be how
   self-driving cars should prioritise the safety of the individuals
   involved in these critical situations. In his piece, Mr. George draws
   attention to a recent effort by the Massachusetts Institute of
   Technology to understand our intuitions on this very issue. MIT's
   project, dubbed `The Moral Machine,' presents us with variations of one
   basic scenario: a self-driving car is in a set of circumstances that
   will inevitably lead to an accident involving both passengers and
   pedestrians. In each variation of this scenario, we are asked to
   indicate whose lives the car should prioritise. These scenarios play
   around with several variables: how many passengers and pedestrians
   there are; whether they are humans or animals; male or female; old or
   young; employed or not; rich or poor; and whether pedestrians are
   jaywalking or not.

   In short, MIT's `Moral Machine' is looking to discern the traits that
   we consider most valuable. The results, so far, show that ceteris
   paribus we tend to prioritise humans over animals; pedestrians over
   passengers; young over old; female over male; employed over unemployed;
   and rich over poor. At first glance, this seems intuitive enough. The
   `Moral Machine' asks us: who should be saved? We answer: those who are
   most useful to society.

   Mr George's piece calls for a "global conversation" on how we are to
   make this calculus, how we are to decide who is more valuable. This is
   seems self-evident: despite the unsurprising general trends that we see
   in the results from the 'Moral Machine,' it is clear that the question
   is not one that elicits a unanimous answer. I will not seek to argue
   for a particular methodology in carrying out this calculus; nor will I
   focus on discrediting the moral foundation of such a calculus -- one
   needn't possess a superior intellect to see that attempting to ascribe
   value to the life of particular individuals is riddled with
   difficulties both moral and practical. Rather, I will argue that to
   propose such a calculus in the context of self-driving cars is wholly
   inappropriate.

   In discussing such a calculus and its inclusion in the algorithms
   governing self-driving cars, we are committing ourselves to two
   alternative conclusions, both of which are troubling.

   The first is that, if we incorporate such a calculus into self-driving
   cars, and in turn equip these cars with the tools necessary to judge,
   in varying situations, the individuals whose lives should be
   prioritised, we are giving these cars moral agency -- the ability to
   make decisions that have a moral dimension. Whether this agency is
   legitimate, of course, is part of the larger debate over all forms of
   artificial intelligence, and whether they should be allowed to have
   human-like privileges. We have no time to engage with this debate here,
   and regardless of what we would conclude, I venture that it is not
   appropriate to allow cars, in essence, to legally decide to save one
   individual at the expense of another -- or, in a different framing, to
   kill one individual in order to save another. Cars should not be making
   moral decisions, and the mere prospect of this possibility would fit so
   poorly within our modern legal system that it would require a
   comprehensive remodelling of our conception of corporate
   individuality.

   The alternative is that we are not, in fact, endowing self-driving cars
   with moral agency. Instead, we are simply incorporating into their
   operating systems a set of guidelines by which they are to judge an
   appropriate course of action -- in other words, a process analogous to
   that of activating breaks when another car is detected in proximity.
   This leads to equally, if not more troubling implications: it would
   mean either that corporations, or, if the global community does engage
   in the debate that Mr. George advocates in his piece, governments, are
   deciding who in a society is valuable, and sanctioning, when it comes
   to it, the sacrifice of those less valuable for the sake of those more
   valuable. This is unacceptable on several grounds.

   First, if we value individual freedom -- and we often congratulate
   ourselves on doing so in the liberal-democratic West -- what amounts to
   being arbitrarily killed is in clear violation of even the weakest
   conceptions of freedom. Second, this evaluation sets a dangerous
   precedent for any number of similar calculuses, and indeed gives
   corporations or governments an alarming degree of control not just over
   how we lead our lives, but also over whether we are entitled to life at
   all. I admit that some might dismiss this further point as a slippery
   slope, but consider this: if we deem it acceptable in certain
   circumstances for corporations or governments to decide who should live
   or die based on their value of society, consistency dictates (in the
   spirit of Kant) that we also deem such decisions acceptable when
   universalised; in other words, if we deem it acceptable for
   corporations or governments to make such decisions, we should also
   accept them on a larger scale -- read: accept genocide carried out for
   socioeconomic reasons, for instance, in cases of marked scarcity. I
   believe we can all agree that genocide is not, in fact, acceptable.

   It follows from this brief discussion, I think, that we should not
   allow self-driving cars to make decisions or judgements that are
   ultimately of a moral character. This is not to say that we should shy
   away from the extraordinary potential that technology has to improve
   wellbeing on a global scale. I am not contesting the fact that
   self-driving cars should, of course, avoid accidents insofar as there
   is freedom to do so; but in circumstances in which an accident is
   inevitable, there should be no moral-evaluative calculus. In these
   circumstances, it cannot be decided which individuals will suffer the
   worst outcomes -- the result must follow, in the first instance, the
   blind guidelines of the law (this is to say, if a younger, more
   `socially useful' pedestrian puts herself in harms way by jaywalking,
   an older, altogether less productive passenger must not be sacrificed);
   and follow, in the second instance, the blind results of chance -- we
   should certainly not take the infamous trolley problem as the blueprint
   for every accident involving self-driving cars.

   Mr. George is right in calling for a global discussion on the future of
   self-driving cars; but this discussion should not be centred on the
   calculus to be carried out by these cars in dire circumstances -- it
   should, instead, be geared toward mitigating the inevitable
   difficulties arising from individual cars with autonomous systems that
   developed by independent companies all operating within the same
   societal framework. There is much space for discussion on this matter,
   but we should all agree on one thing: self-driving cars should not be
   making moral decisions, whether based on our value "preferences" or on
   a simple utilitarian calculus. The results of these decisions would
   essentially amount to sanctioned executions, and this is unacceptable.


   Curious? Try out MIT's \href{http://moralmachine.mit.edu}{'Moral Machine'} for yourselves.

