\addchap{The Most Powerful Climate Lie}

\initial{I}n the wake of the IPCC's stark deadline of a dozen years to pull back from
the brink of climate catastrophe, it is time for immediate action. This
action begins with challenging the most destructive and pervasive lie about
climate change: that it's all on us.

   This month the Intergovernmental Panel for Climate Change (IPCC)
   released their assessment of the consequences of a 1.5 degree rise in
   the Earth's temperature since pre-industrial times, and a 2 degree
   rise. The latter, the agreed limit set by the Paris Agreement three
   years ago, is now known to be a drastic overestimate. Half a degree
   rise from the 1.5 degree threshold means an unprecedented risk to the
   world's most vulnerable people. On our current trajectory, we are set
   to surpass this by 2030. These climate prospects, while bleak, are
   possibly avoidable. The requirement for this is, simply, "rapid,
   far-reaching and unprecedented changes in all aspects of society"
   thereby halving cumulative carbon emissions within the next twelve
   years. Given the context of this call for immediate response by the
   world's leading climate scientists, it is time to revisit our climate
   narrative.

   Media focus on our guilt-ridden, complacent lifestyle choices - from
   the meat industry, to driving, to having more than one child -
   willfully play into the hands of those who possess the real monopoly on
   our climate. The fact that two thirds of all man-made global warming
   emissions have been traced back to 90 companies, highlights just how
   out of sync our focus on the individual is. Carbon emissions are still
   climbing. The answer, so we are told, lies in the individual. If we all
   change our habits for the better we can fix the problem. The problem
   with this is that "we", the individuals, the hegemonic voices of
   mainstream environmentalism, aren't everyone. In the global context,
   "we" are the ones with the money, and the ones doing the most harm. It
   is no coincidence that the richest 10\% are responsible for more than
   half the world's fossil fuel emissions, situated where the scale of
   climate impact is smaller. In the global context, considering the
   geographically unequal effects of climate change, it is the effectivity
   of the richest nations that determines the safety of the environments
   projected to be hit worst. Domestically, the hegemonic focus on ethical
   consumerism as a means of mitigating climate change - from organic
   produce to electric cars - is only feasible for the affluent. The
   economic exclusivity of current environmental solutions means that
   individual actions cannot make the decisive difference needed to tackle
   carbon emissions, it is going to take greater social movements -
   starting at the source.

   As of now, fossil industries are being allowed to continue to quietly
   and diligently profiteer from pushing us closer to the 1.5 degree
   threshold, all the while, fielding no accountability for the
   environmental damages they are inflicting globally. They are also in no
   plan to stop soon. To have hope of staying below a two-degree increase,
   scientists estimate we can pour roughly 565 more gigatons of CO2 into
   the atmosphere by 2050. Most scarily, according to the Carbon Tracker
   Initiative, the carbon (in proven coal, oil and gas reserves) that
   fossil fuel companies intend to release is five times that amount -
   2795 gigatons. According to the same report, is Exxon burns its current
   reserves it would bring us 7\% closer to the 2-degree point. As of now,
   fossil fuel lobbying and corporate donations puppeteering democracy
   have blocked most attempts to limit their impact. All the while the
   burdening the state with the massive task of paying for climate-ready
   infrastructure and picking up the pieces after the ever-increasing
   number of "natural" disasters.

   Media narratives of climate blame are too focused on individual
   responsibility. This is, at best, a misplaced response for the
   immediacy that mitigation requires, at worst, a tool used by industries
   that are working to render those efforts insignificant. The steady and
   slow dismantling of the fossil fuel industry in response to shifts in
   consumer demand is not going to cut it on our updated climate timeline.
   The most immediate way to halt climate change is to hit those who are
   profiteering from the destruction of the planet where it really hurts -
   in the wallet. This will take unprecedented social action put financial
   pressure on oil and gas companies through divestment campaigns,
   corporate accountability efforts and targeting banks and financial
   institutions. On top of this, massive social political movements are
   necessary to regulate corporations, to take back transport, utilities
   and energy grids back into public control, and to raise taxes to fund
   massive investment in climate-ready infrastructure and renewable
   energy. Only in these economic systems will individuals really count -
   when environmental choices are for everyone, not just the affluent.

