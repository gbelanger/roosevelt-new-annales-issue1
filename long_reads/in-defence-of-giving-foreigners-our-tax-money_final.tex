\addchap{In Defence of Giving Foreigners Our Tax Money}
\label{ch:in-defence-of-giving-foreigners-our-tax-money}

{\flushright\largegreycapitals{Dylan Springer}}

\smallskip

   \initial{O}n 22 October U.S. President Donald Trump threatened to cut foreign aid
   to Guatemala, the Honduras, and El Salvador. The justification he
   provided is that a large caravan full of the citizens of these three
   countries is currently headed through Mexico towards the U.S. border.
   The people on the caravan are reportedly fleeing violence and poor
   economic conditions at home, and many hope to find family and a means
   of supporting themselves amongst the sizeable immigrant communities in
   the U.S.

   At the core of Trump's complaint is a simple maxim of common sense -- a
   thought process which goes something like this: `We're sending all
   these people so much money, we are paying for their education, their
   healthcare, their jobs -- and all the while our own people suffer'.
   This idea is not new, or particular to the United States. Noted the BBC
   last year, when the British government announced it was reducing the
   amount of money it spent on foreign aid: `Some people think the UK
   shouldn't be helping people overseas while cutting services at home'.

   I will make three arguments in support of foreign aid. First, we are
   not giving something away and getting nothing in return: foreign aid
   benefits both the sender and the receiver in real, tangible ways.
   Secondly, the money most Western countries do spend on foreign aid is a
   drop in the bucket, fiscally speaking. And, finally, that the little
   our governments do in the name of genuine humanitarianism and altruism
   is worth fighting for.

   In the summer of 1947 George C. Marshall, Secretary of State under U.S.
   President Harry Truman and a distinguished general who had seen action
   in the Philippines, France, and China, made an address at Harvard
   University. The subject was Europe: it was in absolute ruins. The
   recent war had wrecked the Continent's infrastructure, precipitated the
   worst refugee crisis the world had ever seen, and utterly annihilated a
   generation of young men. Worse yet, Marshall said, was the state of its
   economy. For over a decade Europe's nations large and small had been on
   a strong war footing -- every man, woman, and child invested in small
   some measure in the production of armaments and the support of the war
   effort. Now the war was over and the factories were destroyed or,
   unneeded, left vacant. Marshall was proposing that the United States --
   a country which at that time was in possession of half the world's
   total wealth but only 6\% of its population -- should finance Europe's
   recovery. `The remedy lies in... restoring the confidence of the
   European people in the economic future of their own countries and of
   Europe as a whole', he said.

   The American people needed a good reason why they should spend such a
   relatively large amount of cash on a bunch of foreigners, and Marshall
   had one. The plan's purpose, he said, `should be the revival of a
   working economy in the world so as to permit the emergence of political
   and social conditions in which free institutions can exist'. He was
   talking about the Reds. We need Europe to be rich and enjoy
   American-style Western luxuries, he was saying, so that they don't vote
   for Communists. At the time it was a very real threat. Only a few
   months after Marshall made his speech at Harvard, the CIA would soon be
   forced to hand-deliver bags of cash to Italian politicians to stop the
   country from electing a leftist coalition (which was itself funded by
   the Soviets). Like the Marshall Plan, this was foreign aid under
   another name.

   America and its allies have become accustomed to using foreign aid as a
   way to achieve their political goals. Proponents of foreign aid argue
   that the entire world benefits from liberal democratic capitalism,
   especially the biggest and most economically powerful liberal
   democracies (America and its allies). This argument does not convince
   everyone. A lot of people could not care less about the flowering of
   freedom and democracy in all parts of the world. To those people I
   would say that countries like Britain and America actually do get quite
   a lot in return. The countries we send money to let us put our soldiers
   and ships in their cities and ports; and when overall conditions in
   those countries improve, fewer refugees reach our shores. Foreign aid
   helps our P.R., greases the wheels of diplomacy, enhances our national
   security, and advances our national interests. This is why Trump's
   threat to cut aid to Central American countries is counterproductive.
   If he is angry about the caravan of desperate migrants, he should
   consider sending more aid, not less. They would not be fleeing their
   home countries if they were able to lead adequate lives there.

   Furthermore, we should avoid becoming too complacent in the midst of
   Western luxury and hegemony. Liberal democracy is dealing with perhaps
   its greatest threat since the fall of Communism. It faces opposition
   from all corners of the globe. In the Middle East and amongst the
   marginalised communities of the West, radical Islam holds significant
   pull. China has recently been more aggressive in exporting its
   particular brand of technocratic totalitarianism, using its own foreign
   aid and investment programs like the One Belt One Road initiative to
   curry significant power and influence in Africa on the Eurasian
   continent. And, to name just one more example out of many, Europe and
   America are continually fending off a newly-resurgent Russia under the
   extremely capable Vladimir Putin. In short, the logic of the Marshall
   Plan is far from irrelevant today

   It should also be stressed that, in governmental terms, we are talking
   about chump change. Britain spends 0.7\% of its gross national income --
   or about 70 pence for every hundred pounds made in Britain -- on
   foreign aid. And even that is considerably higher than the world
   average. Any reasonable cost-benefit analysis would come down in favour
   of foreign aid.

   Finally, it should not go unsaid that some things are worth doing
   simply because they are right. We who live in advanced, industrialised
   economies like the Britain, America, and Japan are very rich. Those who
   live in developing, agrarian or manufacturing-heavy economies like
   Nigeria, India, and Pakistan are often very poor. Morally and ethically
   speaking, it is our duty to help alleviate their suffering, and, in
   doing so, also help ourselves.

   We live in a totally new era in which it has become -- perhaps for the
   first time, and possibly owing to the extreme interconnectedness of the
   world economy -- genuinely in our interest for nation-states to do the
   right thing. We would be fools not to take advantage of this unique
   moment in history.

%   Sources:
%
%   [21]http://www.bbc.co.uk/newsbeat/article/39653421/uk-foreign-aid-where
%   -does-it-go-and-why
%
%   [22]https://edition.cnn.com/2018/10/18/politics/caravan-foreign-aid/ind
%   ex.html
%
%   [23]http://www.oecd.org/general/themarshallplanspeechatharvarduniversit
%   y5june1947.htm
%
%   [24]https://fullfact.org/economy/uk-spending-foreign-aid/
%
%   [25]https://www.britannica.com/topic/foreign-aid
%
%   [26]https://www.britannica.com/event/Marshall-Plan
%   [27]Dylan SpringerJanuary 30, 2019[28]foreign aid, [29]politics,
