\addchap{The Moral Responsibility in Foreign Aid}
\label{ch:the-moral-responsibility-in-foreign-aid}

{\flushright\largegreycapitals{Archie Batra}}

\smallskip

\initial{D}oes the United Kingdom have a moral responsibility to give foreign aid to
its former colonial possessions?

   The issue (and expense) of foreign aid is often discussed with
   reference to the United Kingdom's erstwhile days as an imperial
   superpower.  One argument that is consistently debated is whether or
   not Britain has a special moral obligation to give foreign aid to its
   former colonial possessions, as some sort of compensation for the time
   that they had to endure British rule. The purpose of this essay is
   therefore to determine the validity of this claim. It seemed
   appropriate to chose India as a case study: Britain's empire was far
   too large to warrant anything other than an analysis of one fragment of
   it, and its prominence within the Empire and the British national
   psyche (i.e. India was the "Jewel in the Crown" of Britain's empire)
   meant that India was a natural choice.  Whilst this does mean that
   conclusions of this essay cannot necessarily be extrapolated, it
   certainly appears that there is no moral imperative demanding that
   Britain give exceptionally generously to states within its old empire.
   This is not to undermine the case for giving foreign aid, but merely to
   say that Britain's former subjects (in this case, India) should not be
   subject to special treatment.

   British rule in India commenced with the passing into law of the 1858
   Government of India Act, which liquidated the British East India
   Company, bestowed all Company property, assets, and responsibilities to
   the Crown, and ended Company Rule in India. The new British Raj would
   rule India until 1947. This period of Indian history has caused much
   debate, with opinion polarised on whether the subcontinent benefitted
   from or was exploited by the British Empire and each conclusion leads
   to different thoughts on the moral aspect to Britain's foreign aid.
   India was undoubtedly changed forever by the British Raj but, to my
   mind, I think to assess whether Crown Rule either benefitted or harmed
   India is ultimately an exercise in futility. The history of Direct Rule
   in India is complex, and the immense amount of evidence that we have
   does not lend itself to reductive or crude labelling as it indicates
   that the British both demonstrably hurt and helped India. Therefore, I
   think that to definitively prove that British colonial rule in India
   did more harm than good (or vice versa) is fundamentally impossible,
   and therefore there is no moral necessity in the United Kingdom's
   foreign aid to India.

   There is, however, a case to be made that British Imperialism inhibited
   the growth of the Indian economy and prosperity on the subcontinent.
   Britain's attitude towards Indian industrial development was lukewarm,
   and largely left to locals. Any development was dictated by Britain's
   own interests, and as such economic growth under the Raj was
   essentially stagnant, rarely exceeding 1\% a year, and negative growth
   was not uncommon. Britain wished to protect their own manufacturing
   interests and encourage their own industrial revolution, and thus India
   experienced large-scale deindustrialisation under the Raj, with India's
   industrial output collapsing from a 25\% share of global output in 1700
   to less than 3\% in 1880. The number of Indians employed in
   manufacturing also greatly declined under British rule, as the share of
   Indians working in manufacturing, mining, and construction fell from
   28.4\% of the workforce in 1881 to just 12.4 in 1911, again implying
   that the British presided over steep industrialisation in India. And,
   not only this, but economist Angus Maddison estimated that India's
   share of global GDP sharply declined from 24.4\% in 1700 to a mere 4.2\%
   in 1950. Jawaharlal Nehru thought that India's economic decline was
   solely a result of British colonial policy, citing tariffs,
   protectionism, and the fact that the British did nothing to help
   nurture Indian industry. It would seem, then, that the British Raj
   oversaw a period of great economic decline in India, did  nothing to
   abate it, and may have even worked to cause it. If this were the whole
   case, it would seem self-evident that Britain owed India money in
   foreign aid.

   However, there is ample evidence to suggest that British colonial rule
   benefited India economically. Firstly, whilst statistics are useful,
   they cannot tell us everything, and certainly tell us nothing of the
   economic life of most Indians. For example, India's large share of
   global GDP in 1700 does not necessarily mean that everyone in India was
   immensely wealthy or had a high standard of living. In fact, the
   British provided the good governance, law, and order over the whole
   subcontinent that enabled Indian trade to grow and facilitated modern
   economic development, as "were [The British] to leave India or Ceylon,
   they would have no customers at all; for, falling into anarchy, they
   would cease at once to export their goods to us and to consume our
   manufactures." The economic history of the Raj is further complicated
   by the fact that the village economy (the sector that represented three
   quarters of the entire population) saw their after-tax income increase
   from 27\% to 54\%, and that by 1914 the British had invested \textsterling\,400 million
   into Indian infrastructure, irrigation, and industry. Further, the
   British has increased the area of irrigated land eight fold. In fact,
   by the 1920s, India was ranked sixth in a table of industrial nations,
   and industrial titans like Jamsetji Tata proved that the British were
   not opposed to Indian industrial development. Thus, the economic legacy
   of the Raj is unclear, and divining the `truth' of India's economy
   impractical. India's economy both benefitted from British rule and was
   harmed by it, and the evidence does not definitively point either way.
   Therefore, there can be no moral imperative behind any money Britain
   decides to give to India now.

   However, we should not solely examine economics: British colonial rule
   in India could also be said to have harmed Indian society as a whole.
   The British presided over some of the worst human tragedies in Indian
   history, including numerous famines that claimed millions of people's
   lives, different episodes of military suppression and massacre, and,
   perhaps most famously, the partition of India in 1947. The Great Famine
   of 1876-78, by way of example, claimed the lives of some estimated over
   four million people across British India. The British, amazingly,
   exported grain from Bombay during this period, resulting in a `grain
   drain' in the region, and thus greatly exacerbating the famine. The
   Amritsar Massacre in 1919 also provides a clear example of British
   colonial rule harming India, as Brigadier-General Reginald Dyer ordered
   his men to open fire of an enclosed crowd of twenty thousand peaceful
   demonstrators, killing up to a thousand people, and only ordering them
   to stop when they ran out of ammunition. Not only this, but the
   British, seeking a quick exit from India, hastily partitioned their
   Indian Empire, and in the process caused a refugee crisis unparalleled
   in history, with communal violence claiming the lives of hundreds upon
   hundreds of thousands of people. This communal violence was compounded
   by the long-standing British policy of `divide and rule', in which
   Muslims and Hindus were pitted against each other in order to make
   India easier to control and govern. All of these atrocities occurred
   under the Raj, and so British colonial rule definitely harmed Indian
   society.

   However, whilst these events were undoubtedly horrible, I think it is
   too crude to say that they were the direct result of British Imperial
   policy, and so the question of whether British colonial rule harmed
   India remains unclear. Partition, for example, was not a simple
   expression of colonial policy, rather the only way that Louis
   Mountbatten could ensure Indian independence without sparking an Indian
   civil war (which would have killed far more people). There were other
   forces that brought about partition, such as Jinnah's absolute refusal
   to consider the possibility of a united India. The Amritsar massacre
   was similarly complex, and owes more to the thoughtless actions of
   Reginald Dyer than any coherent or planned colonial policy. Winston
   Churchill, hardly India's champion, called the Amritsar massacre "an
   extraordinary event, a monstrous event, an event that stands in
   singular and sinister isolation", and Asquith condemned it as "one of
   the worst outrages in the whole of our history." The complexity of
   these tragedies and the incredible variety of factors that caused them
   makes it very difficult to apportion blame (to the Raj or otherwise)
   and again demonstrates that assessing the harm of British colonial rule
   is incredibly difficult. This seems to undermine the moral case for
   Britain giving foreign aid to India in the modern era.

   To complicate the issue further, there is clear evidence that British
   rule benefitted Indian society. An 1895 government report on the
   situation in the North-West frontier of India noted that the local
   population hailed the British as liberators that brought wealth and
   justice to a region that had previously been ruled by tyrants.
   Resolutions of the Indian National Congress show that the British, far
   from wanting to completely oppress and disenfranchise Indians, were
   happy to give them political representation. Indeed, Congress had the
   blessing of the British. The rediscovery and reinvention of Indian
   history and culture was also the direct result of British scholars, who
   wanted to unearth India's rich and hidden past. It was also the British
   colonial government that broke down the archaic barriers of caste,
   religion, and race, which undoubtedly benefitted millions of Indians.
   Paradoxically, Indian nationalism and the unified Indian state could
   not have existed without the Raj; under British rule Indians ceased to
   be merely Bengali, Punjabi, or Tamil, and began to identify as Indian.
   The yearning for freedom and self-determination was based on the
   acceptance of British liberal values, and thus the modern, unified, and
   democratic state of India was created under British auspices. So,
   whilst Indian society undoubtedly suffered at the hands of the British,
   it benefitted as well, and British colonial rule in India is again
   reduced to a crass balance sheet, with the evidence not pointing
   decidedly either way.

   We have an incredible wealth of information about the British Raj, and
   much ink has been spilt on the topic of the British Empire in India.
   However, the evidence is so variegated that it is impossible to make a
   meaningful judgement as to whether British colonial rule either harmed
   or benefited India. Britain didn't even have jurisdiction over a third
   of India (which remained in the hands of the Maharajas) and this
   exemplifies the difficulty in assessing the harm of colonial rule.
   Identifying all of the effects of British rule and attributing them an
   appropriate weight in a cost-benefit analysis of the British Raj is
   just impossible to do.

   Thus, anyone stressing the moral necessity of British foreign aid to
   India is ignoring the complexity of history: there is simply no solid
   base on which to build a `moral' case for foreign aid to India. To
   reiterate points made at the beginning, this is not a case against
   foreign aid as a whole, and this certainly was not an attempt to exempt
   Britain from giving foreign aid. Rather, it was an attempt to
   demonstrate that India and other former colonies do not represent
   special cases on the world stage and that, when considering where the
   foreign aid budget should go, Britain should not prioritise countries
   it once ruled, but should continue look to other objectives such as
   need, efficacy, and consistency with current foreign policy.


%\begin{thebibliography}{dummy}
%\footnotesize
%\bibitem{c:act1858} An Act for the Better Government of India, 21 \& 22 Vict. c. 106, 1858.
%\bibitem{c:fan1895}   H.C. Fanshawe to the Cabinet in 1895, (The National Archives, Catalogue   Ref: CAB 37/39/30.)
%\bibitem{c:clingingsmith2005}   Clingingsmith, D., and Williamson, J. G., India's Deindustrialisation   in the 18th and 19th Centuries (Harvard, 2005).
%\bibitem{c:deshmukh1981}   Deshmukh, Cynthia, `The Bombay International Grain Trade During the   Famine of 1878 in the Bombay Presidency', Proceedings of the Indian   History Congress 42, (1981).
%\bibitem{c:ferguson2004}   Ferguson, Niall, Empire: How Britain Made the World, (London, 2004).
%\bibitem{c:fieldhouse1996}   Fieldhouse, David, `For Richer, for Poorer?' Marshall, P.J.,The   Cambridge Illustrated History of the British Empire, (Cambridge, 1996).
%\bibitem{c:johri2011}   Johri, C.K., India: Perspectives in Politics, Economy, and Labour   1918-2007, Vol 1: The Age of Gandhi, 1918-1957, (Delhi, 2011).
%\bibitem{c:maddison2001}   Maddison, Angus, Development Centre Studies: The World Economy, (Paris,   2001).
%\bibitem{c:obrien1989}   O'Brien, Patrick K., `The Costs and Benefits of British Imperialism   1846-1914: Reply', Past and Present, no. 125, (1989).
%\bibitem]{c:rajan1969}   Rajan, M.S., `The Impact of British Rule in India', Journal of   Contemporary History 4, no. 1, (1969).
%\bibitem{c:robinson1961}   Robinson, R., Gallagher, J., and Denny, A, `Africa and the Victorians: The Official Mind of Imperialism', (London, 1961).
%\bibitem{c:tammita1984}  Tammita-Delgoda, S., A Traveller's History of India,  (Gloucestershire, 1994),
%\bibitem{c:sayer1991}  Sayer, D., `British Reactions to the Amritsar Massacre 1919-1920', Past  and Present, no. 131, (1991).
%\bibitem{c:singh1987}  Singh, Anita Inder, The Origins of the Partition of India 1936-1947,  (Oxford, 1987).
%\bibitem{c:tomlinson2008}  Tomlinson, B.R., The Economy of Modern India, 1860-1970 (Cambridge, 2008).
%\end{thebibliography}
